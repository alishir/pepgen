\section{Introduction}

Decision making is one of the most important
cognitive ability in human beings.  It covers wild
range of situations in human being, for example in
the lower importance level, it affect how people
choose their foods, and in higher level of
importance it affects judgments in courts.
Decision making impairment also related to some
important disease, such as Parkinson, Alzheimer,
and Huntington.

However decision making ability is suspected by
cognitive biases. These cognitive biases affect
decision making process and deviate the process
and cause irrational behavior in humans decisions.
One of the cognitive biases that cause irrational
behavior is framing effect.


Framing effects refers to the fact that context in
which a problem represented biases decision
making.  For example, presenting a problem in
abstract or realistic form could strongly
influence humans answers. Also, representing a
problem in a loss frame or gain frame has a great
effect on the humans choices; both in behavioral
and neurobiological levels.

Framing effect doesn't limited to the rephrasing
the words that represent the problem, it can occur
even, when the problem represented by some simple
figure, without any word or phrase. It also
doesn't limited to single shot decisions and
also could affect sequential decisions. For example if
you ask somebody to split a shape into equal
pieces and incremental make the shapes more
complex, and after some complex shapes ask her to
split a simple square into four equal parts, it
will take more time from her. This is a sample of
framing effect that affect decision making process
response time.

One of the most important phase of sequential
decision making, is learning phase. In the
learning phase peoples learns the value and
consequence of each decision, and try to optimize
their choices to reach the best total outcome.

In this study we analyse the effect of framing
effect on sequential decision making. The task
that selected for sequential decision making is
Iowa Gambling Task, and framing effect induced by
some minor modification in task environment. The
result of Iowa Gambling Task in all of its variant
analysed with a fine granular method. The result
indicates that when subject induced with some
framing effect, there is changes in learning
process, and learning occurs based on gain
frequency instead of long-term outcome.


% vim: textwidth=50
% vim: spell
