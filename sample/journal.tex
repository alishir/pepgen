\documentclass[12pt,twocolumn]{elsarticle}

\usepackage{graphicx}
\usepackage{caption}
\usepackage{subcaption}
\usepackage{hyperref}

\title{Tasks are not Simplex as you Think}
\author{}

\begin{document}
\maketitle
\section{Introduction}
\section{Material and Methods}

We assessed the performance of three
distinct groups of participants in Iowa
gambling task and two variants of it, in
order to found the effect of task
representation in decision making
behavior of participants. Participants'
performance is modeled with a set of
linear equations and a time series based
distance method was used to cluster
participants in each group to find the
effect of task representation on
learning behavior of participants.

% payam section & feature of each deck

Performance of three distinct groups of
participants in Iowa gambling task and
two variants of it evaluated.  The first
variant is the computerized version of
it, and the second one is computerized
version without access to pen and paper.
Demographic information of the
participants in the three tasks is given
in table \ref{table:demographic}.
The distribution of the rewards and
punishments in the three tasks is the
same. The only difference is in the way
of representing the task.  In the first
task, the participants should play a
real card task (RCT). In the first
variant of the Computerized Task (CT),
participants should do the same task,
while it is represented by a simple
computer program not by real cards. In
the Computerized with Pen and Paper Task
(CPPT), the task is represented by the
same computer program and the
participants have access to pen and
paper to freely make notes.  In our
task, similar to Iowa gambling task
subjects are invited to choose on each
trial a card from one of four stationary
decks, named as A, B, C and D. With each
selection, the participant is rewarded
by some points, but every so often a
penalty is also given. In addition to
immediate reward and punishment of each
deck, there is two other features for
each deck that makes each deck unique.
The second feature of each deck is the
relative number of gain vs loss and the
third feature is relative number of net
losses.  Normalized feature weights are
shown in table \ref{table:igt_features}.
Based of each of these three features
decks could be partitioned as
good and bad decks.

\begin{table*}
\caption{Demographic data of the
participants.}
\label{table:demographic}
\centering
\begin{tabular}{cccc}
\textbf{Experiment} & \textbf{No. of
participants} & \textbf{No. of females}
& \textbf{Age range} \\ 
\textbf{Manual} & 32  & 6 & 19-25 \\ 
\textbf{Computerized} & 36  &  9 & 19-25 \\ 
\textbf{Pen \& Paper} & 26  & 5 &  19-25 \\ 
\end{tabular} 
\end{table*}

\begin{table*}
\caption{Normalized weight of features in IGT}
\label{table:igt_features}
\centering
\begin{tabular}{cccc}
\textbf{Deck} & \textbf{Long-term outcome} & \textbf{Gain frequency} & \textbf{Loss frequency} \\ 
\textbf{A} & -0.86  & -0.86 & -1.47 \\ 
\textbf{B} & -0.86  &  0.86 &  0.34 \\ 
\textbf{C} &  0.86  & -0.86 &  0.79 \\ 
\textbf{D} &  0.86  &  0.86 &  0.34 \\ 
\end{tabular} 
\end{table*}

The instructions were given to the
participants verbally before all the
tasks.  The instructions were also
available on the screen at the start
time of the CT and CPPT. In the RCT,
some regular instructions, similar to
those in Bechara et al. are given, But
in the CT, in addition to the regular
instructions, the experimenter told the
participants: "It is important to know
that the computer is not an active agent
in this game. The game is fair. The
computer does not change the order of
the cards. You can imagine that there
are four decks on a table in front of
you." In the CPPT, the instructions were
similar to the CT but the experimenter
added: "You will have a pen and a paper
to write down notes. I want you to
utilize them to increase your points."
In fact, we utilized a common belief
that solving a complicated problem
requires pen and paper, to change
subject belief about task complexity,
and represent the task more complex than
usual.


% linear equation section
Performance of participants was modeled
with a set of linear equations that
described by horstmann et al.. In that
model, for each block of 20 trial, a set
of linear equation $Ax = b$ solved,
where $a_ij$ is the normalized weight of
feature $j$ in deck $i$, $x_j$ in the
corresponding weight of feature $j$, and
$b_i$ is the portion of participant
selection from deck $i$. The result of
solving the given equation is a weight
column $x$ that indicates participant
preference on each feature. By
calculating these weights one can find
the dominant decks feature that
considered by subjects.

% clustering section
In order to analyse learning behavior in
each group, subjects in each group
clustered in to three clusters. These three clusters
are subjects who learned the task,
subjects who have random behavior, and
subjects who didn't learn the task.  The
clustering method deal with subjects
choices in all 100 trial. Before
applying clustering method, subject
choices in 100 trial preprocessed and
converted to good/bad selections across
trials. As mentioned before each deck in
IGT has three distinct features, and
decks could be categorized as good/bad
based on each feature. Each selection
from good decks converted to an upward
movement, and each selection from bad
deck mapped to a downward movement. The
movement step size obtained from
normalization of feature that
categorized decks to good/bad decks. 
The result of
preprocessing step is a time series that
indicates subject behavior across times.
The clustering method deals with these
time series. The similarity measure that
used by clustering algorithm is based on
dynamic time wrapping algorithm. The
main reason for using dynamic time
wrapping algorithms is its ability to
consider time shift in time series. Time
shift consideration is important because
many subjects may have similar behavior
in their choices but with some time
shift, and the clustering algorithm
should put these subjects in to same
cluster. 

% For
% example, a portion of subject original
% selection in IGT, and the result of
% preprocessing based on gain frequency
% show in fig 1. 

% exaple of distance with and without dtw.

To extract subjects with random
behavior, some artificial random
subjects added to each group. Artificial
random subjects, are selection behaviors
that each deck selected with equal
probability in 100 trials. 
Before applying clustering algorithm, some
artificial subject added to each group.
After running
clustering algorithm, a simple
post-processing routine run on obtained
clusters in order to label clusters.
Cluster with maximum number of random
artificial subject labeled as random
cluster and all real subjects that
belong to that cluster treat as subjects
with random behavior. For the other two
clusters, a line fitted to last 60
trials of time series in that cluster.
If the slope of fitted line greater that
0.3 that cluster labeled as learned
otherwise labeled as unlearned.
Schematic of clustering steps is shown
in figure \ref{fig:clustering}.

\begin{figure*}
\caption{Clustering steps}
\label{fig:clustering}
\centering
\includegraphics[scale=0.6]{./img/clustering.png}
\end{figure*}

% static comparison between clustering
% resutl
Two ranked based statical test applied
in order to compare the effect of task
representation on performance of each
group. One statistical test applied
within groups, and the other applied
between groups. Statistical test that
applied within groups, was Friedman
test. The purpose of applied Friedman
test was to identify dominant deck
feature that considered by subjects in
the group. The second statistical test
was between groups. The purpose of this
test was to identify the effect of task
representation on learning behavior of
subjects.  The statistical test that
applied between groups was
Kruskal-Wallis test, because the
possibility of unequal samples from each
group. For each feature of decks, the
clustering algorithm was ran, and then
the Kruskal-Wallis test was applied to
find the effect of task representation
on learning behavior base on selected
feature of decks.


% \section{Results}
several studies have shown that healthy subjects indicates
poor performance on IGT. Usually IGT performance evaluate in
one of the three method follows. The simplest evaluation
method is total money won, another approach is difference
between total advantage and disadvantage selections (net
score), and the third one is pattern of adv/dis selections
across 20 block
trials.


Each deck in IGT has three distinct feature. These features
are, long-term outcome, relative number of gain vs loss and
the relative number of net losses. Combination of these
features is unique for each deck.
Normalized feature weight are shown in table \ref{table:igt_features}.
Decks could be partition into advantage and disadvantage
group with respect to each feature. For example if we
consider only gain frequency feature of decks, decks A and C
are advantage and decks B and D are disadvantage. In this
study we analyse behavior of each subject in a fine
granularity manner. For each feature, we analyse choice by
choice behavior of subjects with respect to advantage/disadvantage decks.
Figure \ref{fig:sample_sub} shows a subject behavior with respect to each three
features, selections from advantage deck corresponds to
upward movements and selections from disadvantage
decks corresponds to downward movement. The magnitude of
movement is based of feature weight that mentioned in table \ref{table:igt_features}. 
As can be seen, most subject choices are base on gain
frequency. 

\begin{table*}
\caption{Normalized weight of features in IGT}
\label{table:igt_features}
\centering
\begin{tabular}{cccc}
\textbf{Deck} & \textbf{Long-term outcome} & \textbf{Gain frequency} & \textbf{Loss frequency} \\ 
\textbf{A} & -0.86  & -0.86 & -1.47 \\ 
\textbf{B} & -0.86  &  0.86 &  0.34 \\ 
\textbf{C} &  0.86  & -0.86 &  0.79 \\ 
\textbf{D} &  0.86  &  0.86 &  0.34 \\ 
\end{tabular} 
\end{table*}


\begin{figure}
\caption{Sample subject performance in different metrics}
\label{fig:sample_sub}
\centering
\includegraphics[scale=0.4]{./img/sample_sub.png}
\end{figure}

Subjects randomly assigned to each variation of IGT.
Relative weight of decks features calculated  by using
linear equation systems model that described in
horstmann2012. In this model for each block of 20 trials a
linear equation system $Ax = b$ solved, where $ a_{ij} $ is the
normalized feature $j$ for deck $i$, and $ b_i $ is the portion of
selections from deck $i$ in 20 trial, and $ x_j $  is relative weight of
feature $j$. Subjects feature weights in each groups varies
significantly as shown in figure \ref{fig:var_in_weights}. In order to find the
source of this considerable variance we visualize subjects
behavior in each group. As mentioned before subject behavior
analysed by trial to trial selections from advantageous and disadvantageous
decks. The criterion that makes a deck advantageous or disadvantageous is
selected feature from deck features (long-term outcome, gain frequency, loss
frequency). 

% one of the interesting pattern that exist in subjects behavior is
% random walk movement pattern in subjects selections.  as can seen in figure 5
% subjects behavior base on long-term outcome, all subjects in PPIGT exhibit
% random behavior.

\begin{figure*}
\caption{Calculated feature weights for each block}
\label{fig:var_in_weights}
\centering
\includegraphics[scale=0.5]{./img/var_in_weights.png}
\end{figure*}

In order to filter out subjects that have random selection
pattern in their choices trial, we add some artificial random
subject to each group. Artificial random subject is a
subject choice pattern in 100 trial that generated randomly
by choosing each deck with equal probability at each trial.
We cluster subjects in each group to three cluster after
adding some artificial subjects to each group. By adding
artificial random subjects and clustering, subjects that
have random walk pattern in their choices join the
artificial random subjects cluster. Two other clusters are
for potential subjects who learned the task and subjects who
did not learn the task.

One of the measures that indicates subjects has been learned
the task is successive selections from advantageous decks.
After running clustering algorithm on subjects choices in
each group, and filtering out the random subjects, we label
the cluster of subjects that have successive selection from
advantageous decks as learned cluster and label the other
cluster as unlearned cluster. In order to find if subject
has successive selections from advantageous decks, we
consider last 60 trials of task and fit a line to it. If the
slope of fitted line is greater than 0.3, we consider that
subject has been learned the task. We notice again that
adv/dis decks depends of feature that we partition decks,
for example if we consider gain-frequency feature and
analyse a subject choice pattern, that subject may learned
the task, however if we consider an other feature the
subject may did not learn the task. In other words learning
the task depends on feature that decks partitioned based on.

In order to compare subjects behavior in different groups
more accurate, for each feature we run clustering algorithm
on subjects. The result of clustering algorithm is three
clusters, randoms subjects, learned subjects and unlearned
subjects. There are some interesting observation after
clustering subjects of each group in these three clusters.
The most interesting observation is that, when we run
clustering base on long-term outcome, nobody in pen \& paper group
(PPIGT) learned the task, and most of the subjects in this group
exhibit random behavior, also some of them did not
learned the task and have successive selections from
disadvantageous decks. However if we choose gain frequency
feature and run clustering based on this feature,
considerable number of subjects fall into learned cluster.
Behavior of subjects in other groups are also interesting,
when run clustering base on long-term outcome learning
occurred in both computerized and manual version of IGT, but number of
learned subjects in manual version is appreciably greater that learned
subjects in computerized version. Also there is no subject in IGT group
that exhibit random behavior, however there is some subject in CPIGT
that exhibit random behavior. Figure \ref{fig:learning} represents the
subset of clustering algorithm result base of long-term outcome and gain
frequency.  In an attempt to compare number of random subjects in each
group, we run clustering algorithms 30 times and count the number of
subjects that fall into random cluster. To find out if number of random
subjects in groups differ significantly we used Kruskal–Wallis statistical test.
The result of Kruskal–Wallis test on number of subjects that fall into random
cluster in 30 runs, indicates that there is a significant difference
between number of subjects that fall into random cluster in PPIGT and
IGT group.

\begin{figure*}
\caption{Subset of Clusters}
\label{fig:learning}
\centering
        \begin{subfigure}[b]{0.4\textwidth}
                \centering
                \includegraphics[width=\textwidth]{./img/outcome_web_learn.png}
                \caption{\tiny{Manual IGT, Learning Cluster}}
                \label{a}
        \end{subfigure}
        \begin{subfigure}[b]{0.4\textwidth}
                \centering
                \includegraphics[width=\textwidth]{./img/outcome_com_learn.png}
                \caption{\tiny{Computerized IGT, Learning Cluster}}
                \label{b}
        \end{subfigure}
        \begin{subfigure}[b]{0.4\textwidth}
                \centering
                \includegraphics[width=\textwidth]{./img/outcome_pen_random.png}
                \caption{\tiny{Pen \& Paper IGT, Random Cluster}}
                \label{c}
        \end{subfigure}
        \begin{subfigure}[b]{0.4\textwidth}
                \centering
                \includegraphics[width=\textwidth]{./img/gain_pen_learn.png}
                \caption{\tiny{Pen \& Paper IGT, Learning Cluster}}
                \label{d}
        \end{subfigure}
\end{figure*}

\section{Conclusion}
\end{document}
