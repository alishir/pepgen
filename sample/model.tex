\documentclass[12pt]{article}

\usepackage{graphicx}
\usepackage{caption}
\usepackage{subcaption}
\usepackage{hyperref}

\title{Simple Decision Making Model for IGT}
\author{}

\begin{document}
\maketitle
\section{Model}

There are lots of variables that influence subject decision even in a simple
task like IGT.  Below is the list of variables that can affect sequential
decision making process:

\begin{itemize}
	\item Age
	\item Sex
	\item IQ
	\item Personality
	\item Mood
	\item Memory
	\item Number of previous decisions
	\item Estimated payoffs
\end{itemize}

In the previous study we found that behavior of subjects spawn over three
different strategies, peoples who maximize their long-term outcome, peoples who
enjoyed gain frequency, and peoples who are loss avers.

In this report, we describe a simple model that includes some of the
aforesaid variables. Variables that included in the current model are:

\begin{itemize}
	\item Some personality traits, reward seeking, fun seeking, loss aversion
	\item Number of previous decisions
	\item Estimated payoffs
\end{itemize}

The model is a simple Bayesian network that span over the time. In the
mathematical notion, decision variable affected by included variables, and we
have the following conditional probability:

\[
	P(D | O, G, L, T, E)
\]

Where $ O, G, L $ are discrete variables that encode intensity of long-term
reward maximization, instant reward affection, and loss aversion traits. $ T $  is
the time that decision has been made and $ E $ is the estimated payoff.


\end{document}

