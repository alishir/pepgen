\documentclass[12pt]{article}

\usepackage{graphicx}
\usepackage{caption}
\usepackage{subcaption}
\usepackage{hyperref}
\usepackage{amsmath}

\title{Simple Decision Making Model for IGT}
\author{}
\date{}

\begin{document}
\maketitle
\section*{Model}

There are lots of variables that influence subject decision even in a simple
task like IGT.  Below is the list of variables that can affect sequential
decision making process:

\begin{itemize}
	\item Age
	\item Sex
	\item IQ
	\item Personality
	\item Mood
	\item Memory
	\item Number of previous decisions
	\item Estimated payoffs
\end{itemize}

In the previous study we found that behavior of subjects spawn over three
different strategies, peoples who maximize their long-term outcome, peoples who
enjoyed gain frequency, and peoples who are loss avers.

In this report, we describe a simple model that includes some of the aforesaid
variables. Variables that included in the current model are:

\begin{itemize}
	\item Some personality traits
	\item Number of previous decisions
	\item Estimated payoffs
\end{itemize}

The model is a simple Bayesian network that span over the time. In the
mathematical notion, decision variable affected by included variables, and we
have the following conditional probability:

\[
	P(D | O, G, L, T, E)
\]

Where $ O, G, L $ are discrete variables that encode intensity of long-term
reward maximization, instant reward affection, and loss aversion traits. $ T $
is the time that decision has been made and $ E $ is the estimated payoff.

We assume that $ O, G, L $ are independent from each other but all of them
dependent on $E$, also we assume that $ T $ is independent from all other
variables. In summary:

\[
	P(D | O, G, L, T, E) = P(D | O, E) P(D | G, E) P(D | L, E) P(D | T)
\]

We also assume that the previous decision doesn't influence current decision,
but the result of previous decision updates the value of estimated payoff, $ E
$. Figure \ref{fig:bn_model} shown the relation between variables.

\begin{figure}[b]
\caption{Relation between variables in model}
\label{fig:bn_model}
\centering
\includegraphics[scale=0.4]{./img/bn_model.png}
\end{figure}

The last things about the model are the decision making mechanism and estimated
payoff calculation method. Estimated payoff calculation is based on average and
normalization, it aggregates previous reward and punishments, and extract each
feature from them.  Basis of decision making mechanism is epsilon greedy
decision making method, with some modification to consider time effect.

\[
	P(D | O, G, L, T, E) = \epsilon-Greedy(E_{4 \times 3} \times [O, G, L]^{'}) exp(T/T^{*})
\]

Where $\epsilon-Greedy([l_1, \dots, l_n])$ calculate $\dfrac{e^{l_j}}{\sum_{i = 1}^{n} e^{l_i}}$ for each $l_j$, 
and $T^{*}$ is the total time of the IGT game, here is $100$.
The result of $E_{4 \times 3} \times [O, G, L]^{'}$ is the subjective aggregated value for each deck.
$E_{4 \times 3}$ is a matrix that encode estimated value of each feature for each deck, and $[O, G, L]$ is the 
importance of each feature.


\end{document}

