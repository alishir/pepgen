\section{Results}
several studies have shown that healthy subjects indicates
poor performance on IGT. Usually IGT performance evaluate in
one of the three method follows. The simplest evaluation
method is total money won, another approach is difference
between total advantage and disadvantage selections (net
score), and the third one is pattern of adv/dis selections
across 20 block
trials.


Each deck in IGT has three distinct feature. These features
are, long-term outcome, relative number of gain vs loss and
the relative number of net losses. Combination of these
features is unique for each deck.
Normalized feature weight are shown in table \ref{table:igt_features}.
Decks could be partition into advantage and disadvantage
group with respect to each feature. For example if we
consider only gain frequency feature of decks, decks A and C
are advantage and decks B and D are disadvantage. In this
study we analyse behavior of each subject in a fine
granularity manner. For each feature, we analyse choice by
choice behavior of subjects with respect to advantage/disadvantage decks.
Figure \ref{fig:sample_sub} shows a subject behavior with respect to each three
features, selections from advantage deck corresponds to
upward movements and selections from disadvantage
decks corresponds to downward movement. The magnitude of
movement is based of feature weight that mentioned in table \ref{table:igt_features}. 
As can be seen, most subject choices are base on gain
frequency. 

\begin{table*}
\caption{Normalized weight of features in IGT}
\label{table:igt_features}
\centering
\begin{tabular}{cccc}
\textbf{Deck} & \textbf{Long-term outcome} & \textbf{Gain frequency} & \textbf{Loss frequency} \\ 
\textbf{A} & -0.86  & -0.86 & -1.47 \\ 
\textbf{B} & -0.86  &  0.86 &  0.34 \\ 
\textbf{C} &  0.86  & -0.86 &  0.79 \\ 
\textbf{D} &  0.86  &  0.86 &  0.34 \\ 
\end{tabular} 
\end{table*}


\begin{figure}
\caption{Sample subject performance in different metrics}
\label{fig:sample_sub}
\centering
\includegraphics[scale=0.4]{./img/sample_sub.png}
\end{figure}

Subjects randomly assigned to each variation of IGT.
Relative weight of decks features calculated  by using
linear equation systems model that described in
horstmann2012. In this model for each block of 20 trials a
linear equation system $Ax = b$ solved, where $ a_{ij} $ is the
normalized feature $j$ for deck $i$, and $ b_i $ is the portion of
selections from deck $i$ in 20 trial, and $ x_j $  is relative weight of
feature $j$. Subjects feature weights in each groups varies
significantly as shown in figure \ref{fig:var_in_weights}. In order to find the
source of this considerable variance we visualize subjects
behavior in each group. As mentioned before subject behavior
analysed by trial to trial selections from advantageous and disadvantageous
decks. The criterion that makes a deck advantageous or disadvantageous is
selected feature from deck features (long-term outcome, gain frequency, loss
frequency). 

% one of the interesting pattern that exist in subjects behavior is
% random walk movement pattern in subjects selections.  as can seen in figure 5
% subjects behavior base on long-term outcome, all subjects in PPIGT exhibit
% random behavior.

\begin{figure*}
\caption{Calculated feature weights for each block}
\label{fig:var_in_weights}
\centering
\includegraphics[scale=0.5]{./img/var_in_weights.png}
\end{figure*}

In order to filter out subjects that have random selection
pattern in their choices trial, we add some artificial random
subject to each group. Artificial random subject is a
subject choice pattern in 100 trial that generated randomly
by choosing each deck with equal probability at each trial.
We cluster subjects in each group to three cluster after
adding some artificial subjects to each group. By adding
artificial random subjects and clustering, subjects that
have random walk pattern in their choices join the
artificial random subjects cluster. Two other clusters are
for potential subjects who learned the task and subjects who
did not learn the task.

One of the measures that indicates subjects has been learned
the task is successive selections from advantageous decks.
After running clustering algorithm on subjects choices in
each group, and filtering out the random subjects, we label
the cluster of subjects that have successive selection from
advantageous decks as learned cluster and label the other
cluster as unlearned cluster. In order to find if subject
has successive selections from advantageous decks, we
consider last 60 trials of task and fit a line to it. If the
slope of fitted line is greater than 0.3, we consider that
subject has been learned the task. We notice again that
adv/dis decks depends of feature that we partition decks,
for example if we consider gain-frequency feature and
analyse a subject choice pattern, that subject may learned
the task, however if we consider an other feature the
subject may did not learn the task. In other words learning
the task depends on feature that decks partitioned based on.

In order to compare subjects behavior in different groups
more accurate, for each feature we run clustering algorithm
on subjects. The result of clustering algorithm is three
clusters, randoms subjects, learned subjects and unlearned
subjects. There are some interesting observation after
clustering subjects of each group in these three clusters.
The most interesting observation is that, when we run
clustering base on long-term outcome, nobody in pen \& paper group
(PPIGT) learned the task, and most of the subjects in this group
exhibit random behavior, also some of them did not
learned the task and have successive selections from
disadvantageous decks. However if we choose gain frequency
feature and run clustering based on this feature,
considerable number of subjects fall into learned cluster.
Behavior of subjects in other groups are also interesting,
when run clustering base on long-term outcome learning
occurred in both computerized and manual version of IGT, but number of
learned subjects in manual version is appreciably greater that learned
subjects in computerized version. Also there is no subject in IGT group
that exhibit random behavior, however there is some subject in CPIGT
that exhibit random behavior. Figure \ref{fig:learning} represents the
subset of clustering algorithm result base of long-term outcome and gain
frequency.  In an attempt to compare number of random subjects in each
group, we run clustering algorithms 30 times and count the number of
subjects that fall into random cluster. To find out if number of random
subjects in groups differ significantly we used Kruskal–Wallis statistical test.
The result of Kruskal–Wallis test on number of subjects that fall into random
cluster in 30 runs, indicates that there is a significant difference
between number of subjects that fall into random cluster in PPIGT and
IGT group.

\begin{figure*}
\caption{Subset of Clusters}
\label{fig:learning}
\centering
        \begin{subfigure}[b]{0.4\textwidth}
                \centering
                \includegraphics[width=\textwidth]{./img/outcome_web_learn.png}
                \caption{\tiny{Manual IGT, Learning Cluster}}
                \label{a}
        \end{subfigure}
        \begin{subfigure}[b]{0.4\textwidth}
                \centering
                \includegraphics[width=\textwidth]{./img/outcome_com_learn.png}
                \caption{\tiny{Computerized IGT, Learning Cluster}}
                \label{b}
        \end{subfigure}
        \begin{subfigure}[b]{0.4\textwidth}
                \centering
                \includegraphics[width=\textwidth]{./img/outcome_pen_random.png}
                \caption{\tiny{Pen \& Paper IGT, Random Cluster}}
                \label{c}
        \end{subfigure}
        \begin{subfigure}[b]{0.4\textwidth}
                \centering
                \includegraphics[width=\textwidth]{./img/gain_pen_learn.png}
                \caption{\tiny{Pen \& Paper IGT, Learning Cluster}}
                \label{d}
        \end{subfigure}
\end{figure*}
